%% LyX 2.3.7 created this file.  For more info, see http://www.lyx.org/.
%% Do not edit unless you really know what you are doing.
\documentclass{scrartcl}
\usepackage[T1]{fontenc}
\usepackage{array}
\usepackage{polyglossia}
\usepackage{luacode}
\setdefaultlanguage{german}

%%InvoiceRecipient =
%%    {
%%    name = "Max Mustermann",
%%    street = "Musterstraße 17",
%%    city = "Musterstadt",
%%    postCode = "12345",
%%    country = "DE"
%%}

\begin{luacode}
  invoiceRecipient = {
    name = "Max Mustermann",
    street = "Musterstraße 17",
    city = "Musterstadt",
    postCode = "12345",
    country = "DE"
  }

  invoiceItems = {
    [1] = {
      description = "vfdsdsfdsfdsfs fdsfdskfdsk fdskfk fkwef kefkwekfe ANS 23054303053",
      quantity = 2,
      netAmount = 3.35,
      taxation = 19,
      totalAmount = 3.99,
      sum = 7.98,
      currency = "EUR"
    },
    [2] = {
      description = "vf ds dsf dsf dsfs fds fd skf dsk ANS 606406540",
      quantity = 1,
      netAmount = 9.07,
      taxation = 19,
      totalAmount = 10.79,
      sum = 10.79,
      currency = "EUR"
    },
    [3] = {
      description = "Versandkosten",
      quantity = 1,
      netAmount = 0.00,
      taxation = 0,
      totalAmount = 0.00,
      sum = 0.00,
      currency = "EUR"
    },
  }

\end{luacode}

\makeatletter

%%%%%%%%%%%%%%%%%%%%%%%%%%%%%% LyX specific LaTeX commands.
%% Because html converters don't know tabularnewline
\providecommand{\tabularnewline}{\\}

\makeatother

\begin{document}
\begin{minipage}[t]{0.4\columnwidth}%
\begin{flushleft}
restaurant.think.different
\par\end{flushleft}%
\end{minipage} \hfill{}%
\begin{minipage}[t]{0.4\columnwidth}%
\begin{flushright}
Rechnung
\par\end{flushright}%
\end{minipage}

\bigskip{}

\begin{minipage}[t]{0.4\columnwidth}
\directlua{tex.print(invoiceRecipient.name)}

\directlua{tex.print(invoiceRecipient.street)}

\directlua{tex.print(invoiceRecipient.city)}, \directlua{tex.print(invoiceRecipient.postCode)}

\directlua{tex.print(invoiceRecipient.country)}
\end{minipage}
\hfill{}%
\begin{minipage}[t]{0.4\columnwidth}%
{Referenznummer XXX}{\par}

{Verkauf durch Verkäufer}{\par}

{\rule[0.5ex]{1\columnwidth}{1pt}}{\par}

{Rechnungsdatum /}\\
{Lieferdatum 4.9.65}{\par}%
\end{minipage}

\vspace{10ex}

Rechnungsdetails

\hfill{}%
\begin{minipage}[t]{1\columnwidth}%
    \begin{luacode}
    tex.print("\\begin{tabular*}{0.99\\columnwidth}{@{\\extracolsep{\\fill}}>{\\raggedright}p{0.3\\columnwidth}rrrrr}")
    tex.print("{Beschreibung} & {Menge} & {Stückpreis} & {USt.\\%} & {Stückpreis} & {Summe}\\tabularnewline")
    for i, item in ipairs(invoiceItems) do
      tex.sprint("{" ..  item.description .. "}")
      tex.sprint(" & " )
      tex.sprint("{" ..  item.quantity .. "}")
      tex.sprint(" & " )
      tex.sprint("{" ..  item.netAmount .. " €}")
      tex.sprint(" & " )
      tex.sprint("{" ..  item.taxation .. " \\%}")
      tex.sprint(" & " )
      tex.sprint("{" ..  item.totalAmount .. " €}")
      tex.sprint(" & " )
      tex.sprint("{" ..  item.sum .. " €}")
      tex.print("\\tabularnewline")
    end
    tex.print("\\end{tabular*}")
  \end{luacode}
\end{minipage}

\hfill{}%
\begin{minipage}[t]{0.3\columnwidth}%
Gesamtpreis%
\end{minipage}%
\noindent\begin{minipage}[t]{0.1\columnwidth}%
\begin{flushright}
18,77 €
\end{flushright}%
\end{minipage}
\end{document}
